% Options for packages loaded elsewhere
\PassOptionsToPackage{unicode}{hyperref}
\PassOptionsToPackage{hyphens}{url}
%
\documentclass[
]{book}
\usepackage{amsmath,amssymb}
\usepackage{lmodern}
\usepackage{ifxetex,ifluatex}
\ifnum 0\ifxetex 1\fi\ifluatex 1\fi=0 % if pdftex
  \usepackage[T1]{fontenc}
  \usepackage[utf8]{inputenc}
  \usepackage{textcomp} % provide euro and other symbols
\else % if luatex or xetex
  \usepackage{unicode-math}
  \defaultfontfeatures{Scale=MatchLowercase}
  \defaultfontfeatures[\rmfamily]{Ligatures=TeX,Scale=1}
\fi
% Use upquote if available, for straight quotes in verbatim environments
\IfFileExists{upquote.sty}{\usepackage{upquote}}{}
\IfFileExists{microtype.sty}{% use microtype if available
  \usepackage[]{microtype}
  \UseMicrotypeSet[protrusion]{basicmath} % disable protrusion for tt fonts
}{}
\makeatletter
\@ifundefined{KOMAClassName}{% if non-KOMA class
  \IfFileExists{parskip.sty}{%
    \usepackage{parskip}
  }{% else
    \setlength{\parindent}{0pt}
    \setlength{\parskip}{6pt plus 2pt minus 1pt}}
}{% if KOMA class
  \KOMAoptions{parskip=half}}
\makeatother
\usepackage{xcolor}
\IfFileExists{xurl.sty}{\usepackage{xurl}}{} % add URL line breaks if available
\IfFileExists{bookmark.sty}{\usepackage{bookmark}}{\usepackage{hyperref}}
\hypersetup{
  pdftitle={Cloud-SPAN Handbook},
  hidelinks,
  pdfcreator={LaTeX via pandoc}}
\urlstyle{same} % disable monospaced font for URLs
\usepackage{longtable,booktabs,array}
\usepackage{calc} % for calculating minipage widths
% Correct order of tables after \paragraph or \subparagraph
\usepackage{etoolbox}
\makeatletter
\patchcmd\longtable{\par}{\if@noskipsec\mbox{}\fi\par}{}{}
\makeatother
% Allow footnotes in longtable head/foot
\IfFileExists{footnotehyper.sty}{\usepackage{footnotehyper}}{\usepackage{footnote}}
\makesavenoteenv{longtable}
\usepackage{graphicx}
\makeatletter
\def\maxwidth{\ifdim\Gin@nat@width>\linewidth\linewidth\else\Gin@nat@width\fi}
\def\maxheight{\ifdim\Gin@nat@height>\textheight\textheight\else\Gin@nat@height\fi}
\makeatother
% Scale images if necessary, so that they will not overflow the page
% margins by default, and it is still possible to overwrite the defaults
% using explicit options in \includegraphics[width, height, ...]{}
\setkeys{Gin}{width=\maxwidth,height=\maxheight,keepaspectratio}
% Set default figure placement to htbp
\makeatletter
\def\fps@figure{htbp}
\makeatother
\setlength{\emergencystretch}{3em} % prevent overfull lines
\providecommand{\tightlist}{%
  \setlength{\itemsep}{0pt}\setlength{\parskip}{0pt}}
\setcounter{secnumdepth}{5}
\usepackage{booktabs}
\ifluatex
  \usepackage{selnolig}  % disable illegal ligatures
\fi
\usepackage[]{natbib}
\bibliographystyle{apalike}

\title{Cloud-SPAN Handbook}
\author{}
\date{\vspace{-2.5em}}

\begin{document}
\maketitle

{
\setcounter{tocdepth}{1}
\tableofcontents
}
\hypertarget{preface}{%
\chapter*{Preface}\label{preface}}
\addcontentsline{toc}{chapter}{Preface}

\includegraphics{images/cloud-span-logo-text.png}

This is the handbook for University of York Biology Department's \href{https://cloud-span.york.ac.uk/}{Cloud-SPAN} project.

\hypertarget{handbook-contents}{%
\section*{Handbook Contents 📋}\label{handbook-contents}}
\addcontentsline{toc}{section}{Handbook Contents 📋}

⭐\href{https://cloud-span.github.io/CloudSPAN-handbook/introduction.html}{Introduction}\\
🤝\href{https://cloud-span.github.io/CloudSPAN-handbook/code-of-conduct.html}{Code of Conduct}\\
👪\href{https://cloud-span.github.io/CloudSPAN-handbook/the-cloud-span-commmunity.html}{The Cloud-SPAN Community}\\
📌\href{https://cloud-span.github.io/CloudSPAN-handbook/fair-principles.html}{FAIR Principles}\\
📜\href{https://cloud-span.github.io/CloudSPAN-handbook/forum.html}{Cloud-SPAN Online Forum}

\hypertarget{introduction}{%
\chapter{Introduction}\label{introduction}}

Welcome to the Cloud-SPAN Community Handbook! It's great to see you here.

\hypertarget{about-cloud-span}{%
\section{About Cloud-SPAN}\label{about-cloud-span}}

Cloud-SPAN deploys high quality learning resources that will train researchers to effectively generate and analyse a range of 'omics data using Cloud computing resources.

Cloud-SPAN is a collaboration between the \href{https://www.york.ac.uk/biology/}{Department of Biology} at the University of York and the \href{https://www.software.ac.uk/}{Software Sustainability Institute}, and funded by the \href{https://www.ukri.org/news/initiatives-boost-health-and-bioscience-skills-and-industry/}{UKRI innovation scholars award} under project reference \href{https://gtr.ukri.org/projects?ref=MR\%2FV038680\%2F1}{MR/V038680/1}.

\hypertarget{about-this-handbook}{%
\section{About this handbook}\label{about-this-handbook}}

This handbook is intended as a reference for both the core Cloud-SPAN team (see below) and for our wider community of learners. It's where you'll find our \protect\hyperlink{code-of-conduct}{Code of Conduct}, contributing guidelines and other practical information which will help you make the most of our resources in a friendly, understanding environment.

For example, did you know that we have an online forum for questions, suggestions and any concerns you might have about applying what you learn on our courses? We strongly encourage you to engage with the Cloud-SPAN community to enhance your learning and understanding. You can find the forum \href{cloudspan.peerboard.com}{here} or read more about \protect\hyperlink{forum}{using Peerboard} later in this manual.

\hypertarget{our-team}{%
\section{Our Team}\label{our-team}}

\hypertarget{project-team}{%
\subsection{Project team}\label{project-team}}

\begin{longtable}[]{@{}
  >{\raggedright\arraybackslash}p{(\columnwidth - 8\tabcolsep) * \real{0.20}}
  >{\raggedright\arraybackslash}p{(\columnwidth - 8\tabcolsep) * \real{0.20}}
  >{\raggedright\arraybackslash}p{(\columnwidth - 8\tabcolsep) * \real{0.20}}
  >{\raggedright\arraybackslash}p{(\columnwidth - 8\tabcolsep) * \real{0.20}}
  >{\raggedright\arraybackslash}p{(\columnwidth - 8\tabcolsep) * \real{0.20}}@{}}
\toprule
Name & Role & \% working on the project & Time period allocated to the project & Institution \\
\midrule
\endhead
Emma Rand & Project oversight & 20\% & 2020-22 & Uni of York \\
James Chong & Project oversight & 10\% & 2020-22 & Uni of York \\
Jorge Buenabad-Chavez & Content developer and Cloud deliverer & 100\% & 2020-22 & Uni of York \\
Evelyn Greeves & Content developer and FAIR/CoP deliverer & 60\% & 2020-22 & Uni of York \\
Sarah Dowsland & Project Manager & 50\% & 2020-22 & Uni of York \\
Annabel Cansdale & Content developer & 20\% & 2020-22 & Uni of York \\
Sarah Forrester & Content developer & 20\% & 2020-22 & Uni of York \\
Neil Chue Hong & Strategic advisor & 10\% & 2020-22 & Software Sustainability Institute \\
Emma Barnes & Project oversight & 20\% & 2020-22 & Uni of York \\
Dan Bishop & Strategic input & 10\% & 2020-22 & Uni of York \\
\bottomrule
\end{longtable}

\hypertarget{code-of-conduct}{%
\chapter{Code of Conduct}\label{code-of-conduct}}

\hypertarget{preamble}{%
\section{Preamble}\label{preamble}}

The CloudSPAN team are dedicated to providing a welcoming and supportive environment for all people, regardless of background or identity. As such, we do not tolerate behaviour that is disrespectful to our community members or that excludes, intimidates, or causes discomfort to others. We do not tolerate discrimination or harassment based on characteristics that include, but are not limited to: gender identity and expression, sexual orientation, disability, physical appearance, body size, citizenship, nationality, ethnic or social origin, pregnancy, familial status, veteran status, genetic information, religion or belief (or lack thereof), membership of a national minority, property, age, education, socio-economic status, technical choices, and experience level.

Everyone who participates in CloudSPAN project activities is required to conform to this Code of Conduct. This Code of Conduct applies to all spaces managed by the CloudSPAN project including, but not limited to, in person focus groups and workshops, and communications online via GitHub. By participating, contributors indicate their acceptance of the procedures by which the project core development team resolves any Code of Conduct incidents, which may include storage and processing of their personal information.

\hypertarget{our-code-of-conduct}{%
\section{Our Code of Conduct}\label{our-code-of-conduct}}

We are confident that our community members will together build a supportive and collaborative atmosphere at our events and during online communications. The following bullet points set out explicitly what we hope you will consider to be appropriate community guidelines:

\begin{itemize}
\item
  \textbf{Be respectful of different viewpoints and experiences.} Do not engage in homophobic, racist, transphobic, ageist, ableist, sexist, or otherwise exclusionary behaviour.
\item
  \textbf{Use welcoming and inclusive language.} Exclusionary comments or jokes, threats or violent language are not acceptable. Do not address others in an angry, intimidating, or demeaning manner. Be considerate of the ways the words you choose may impact others. Be patient and respectful of the fact that English is a second (or third or fourth!) language for some participants.
\item
  \textbf{Do not harass people.} Harassment includes unwanted physical contact, sexual attention, or repeated social contact (see below for an extended list of behaviours we consider to be harassment). Know that consent is explicit, conscious and continuous---not implied. If you are unsure whether your behaviour towards another person is welcome, ask them. If someone tells you to stop, do so.
\item
  \textbf{Respect the privacy and safety of others.} Do not take photographs of others without their permission. Do not share other participant's personal experiences without their express permission. Note that posting (or threatening to post) personally identifying information of others without their consent (``doxing'') is a form of harassment.
\item
  \textbf{Be considerate of others' participation.} Everyone should have an opportunity to be heard. In update sessions, please keep comments succinct so as to allow maximum engagement by all participants. Do not interrupt others on the basis of disagreement; hold such comments until they have finished speaking.
\item
  \textbf{Don't be a bystander.} If you see something inappropriate happening, speak up. If you don't feel comfortable intervening but feel someone should, please feel free to ask a member of the Code of Conduct response team for support.
\item
  As an overriding general rule, \textbf{please be intentional in your actions and humble in your mistakes}.
\end{itemize}

All interactions should be professional regardless of platform: either online or in-person. See \href{https://www.recurse.com/manual\#sub-sec-social-rules}{this explanation of the four social rules} - no feigning surprise, no well-actually's, no back-seat driving, no subtle -isms - for further recommendations for inclusive behaviours.

\hypertarget{unacceptable-behaviour}{%
\subsection{Unacceptable Behaviour}\label{unacceptable-behaviour}}

Examples of unacceptable behaviour by community members at any project event or platform include:

\begin{itemize}
\tightlist
\item
  written or verbal comments which have the effect of excluding people on the basis of membership of any specific group
\item
  causing someone to fear for their safety, such as through stalking, following, or intimidation
\item
  violent threats or language directed against another person
\item
  the display of sexual or violent images
\item
  unwelcome sexual attention
\item
  nonconsensual or unwelcome physical contact
\item
  sustained disruption of talks, events or communications
\item
  insults or put downs
\item
  sexist, racist, homophobic, transphobic, ableist, or exclusionary jokes
\item
  excessive swearing
\item
  incitement to violence, suicide, or self-harm
\item
  continuing to initiate interaction (including photography or recording) with someone after being asked to stop
\item
  publication of private communication without consent
\end{itemize}

CloudSPAN prioritises marginalised people's safety over privileged people's comfort. We will not act on complaints regarding:

\begin{itemize}
\tightlist
\item
  `Reverse' -isms, including `reverse racism,' `reverse sexism,' and `cisphobia'.
\item
  Reasonable communication of boundaries, such as ``leave me alone,'' ``go away,'' or ``I'm not discussing this with you.''
\item
  Communicating in a `tone' you don't find congenial.
\item
  Criticism of racist, sexist, cissexist, or otherwise oppressive behavior or assumptions.
\end{itemize}

\hypertarget{incident-reporting-guidelines}{%
\section{Incident Reporting Guidelines}\label{incident-reporting-guidelines}}

\hypertarget{contact-points}{%
\subsection{Contact points}\label{contact-points}}

If you feel able to, please contact Emma Rand by email at \href{mailto:emma.rand@york.ac.uk}{\nolinkurl{emma.rand@york.ac.uk}}

\hypertarget{alternate-contact-points}{%
\subsection{Alternate contact points}\label{alternate-contact-points}}

If you do not feel comfortable contacting Emma Rand, please report an incident to Evelyn Greeves by email at \href{mailto:evelyn.greeves@york.ac.uk}{\nolinkurl{evelyn.greeves@york.ac.uk}}

\hypertarget{acknowledgements}{%
\section{Acknowledgements}\label{acknowledgements}}

This Code was adapted from the \href{https://the-turing-way.netlify.app/community-handbook/coc.html}{Turing Way} Code of Conduct, which itself draws from the \href{https://docs.carpentries.org/topic_folders/policies/code-of-conduct.html}{Carpentries} and \href{https://docs.google.com/document/d/1iv2cizNPUwtEhHqaezAzjIoKkaIX02f7XbYmFMXDTGY/edit\#heading=h.eawfypsf8ylf}{Alan Turing Institute Data Study Group} codes of conduct. Both are licensed for reuse under a \href{https://creativecommons.org/licenses/by/4.0/}{CC BY 4.0 CA} license.

Material was additionally drawn from the \href{https://github.com/RConsortium/RCDI-WG/blob/0ca0a91dccc9296ff53a5806f52a2a49dbb8850d/conduct/code-of-conductd}{R Community Diversity, Equity, and Inclusion Working Group}, also licensed under \href{https://creativecommons.org/licenses/by/4.0/}{CC BY 4.0 CA}.

\hypertarget{the-cloud-span-commmunity}{%
\chapter{The Cloud-SPAN commmunity}\label{the-cloud-span-commmunity}}

⚠️ \textbf{Under construction} ⚠️

Our aim is to build a friendly and involved community of people who have used our resources, are interested in our resources, or who have expertise in the areas we cover.

That means that whether you are\ldots{}\\

🧬 an expert in 'omics analyses\\
🤔 a complete newbie at 'omics\\
🤷 not entirely sure what an ``omic'' is\\
☁️ an experienced Cloud user\\
😵 a little bewildered by the Cloud\\

\ldots then the Cloud-SPAN community is for you!

There are lots of ways to contribute and we welcome all of them! Contributing, and joining our community, doesn't have to mean writing technical code.

Here are some ideas of ways you can contribute:

\hypertarget{ways-to-contribute}{%
\section{Ways to contribute}\label{ways-to-contribute}}

\hypertarget{learn}{%
\subsection*{Learn}\label{learn}}
\addcontentsline{toc}{subsection}{Learn}

\begin{itemize}
\tightlist
\item
  Attend or work through our Foundations in Genomics course.
\item
  Ask questions on our \href{https://cloudspan.peerboard.com/}{community forum}.
\end{itemize}

\hypertarget{connect}{%
\subsection*{Connect}\label{connect}}
\addcontentsline{toc}{subsection}{Connect}

\begin{itemize}
\tightlist
\item
  Join our \emph{Community of Practice}.
\end{itemize}

\hypertarget{help}{%
\subsection*{Help}\label{help}}
\addcontentsline{toc}{subsection}{Help}

\begin{itemize}
\tightlist
\item
  Answer questions on our \href{https://cloudspan.peerboard.com/}{community forum}.
\item
  Tell us about bugs or problems you encounter in the course.
\end{itemize}

\hypertarget{expand}{%
\subsection*{Expand}\label{expand}}
\addcontentsline{toc}{subsection}{Expand}

\begin{itemize}
\tightlist
\item
  Suggest new/different software tools for analysis.
\item
  Contribute new examples.
\item
  Attend one of our 'Train the Trainer" courses so you can take parts of the course back to teach at your home institution.
\end{itemize}

\hypertarget{using-github-to-contribute}{%
\section{Using GitHub to contribute}\label{using-github-to-contribute}}

We use GitHub as a tool for managing version control (AKA keeping a record of the project's development). This helps us stay accountable and transparent. It's also one of the ways we are making steps towards adhering to the \protect\hyperlink{fair-principles}{FAIR Principles}.

If you want to contribute any content such as an update to the course or a new example then via GitHub is the best place to get in contact. For lots more guidance about how to contribute via GitHub, read our \href{https://github.com/Cloud-SPAN/CloudSPAN-handbook/blob/main/CONTRIBUTING.md}{GitHub Contribution Guide}.

Git (the programming language underlying Github) and Github can be a little intimidating at first but don't worry, the team are here to hold your hand! 🤝

\hypertarget{fair-principles}{%
\chapter{FAIR Principles}\label{fair-principles}}

\hypertarget{what-is-fair-data}{%
\section{What is FAIR data?}\label{what-is-fair-data}}

FAIR data is \textbf{Findable}, \textbf{Accessible}, \textbf{Interoperable} and \textbf{Reusable}.

These principles are designed to help both humans and machines find and reuse data as easily as possible. They are aspirational but tangible steps can be made towards realising them.

You can read about the ethical values underlying the FAIR principles via the FAIR Cookbook \href{https://fairplus.github.io/the-fair-cookbook/content/recipes/introduction/FAIRplus-values.html}{here}.

\hypertarget{findable}{%
\subsection*{Findable}\label{findable}}
\addcontentsline{toc}{subsection}{Findable}

\textbf{Findable} is all about making sure data/resources are as easy to find as possible.

How we're making resources \textbf{findable} at Cloud-SPAN:

\begin{itemize}
\tightlist
\item
  We will be assigning persistent identifiers to our teaching materials to prevent ``link rot'', or broken links.
\item
  We will register our teaching materials with an appropriate registry (e.g.~Carpentries Incubator), so they are easier to find.
\item
  We will be describing all our resources with rich metadata so they can be aggregated by the right registries.
\end{itemize}

\hypertarget{accessible}{%
\subsection*{Accessible}\label{accessible}}
\addcontentsline{toc}{subsection}{Accessible}

\textbf{Accessible} means it is easy to find out how to access the data/resources.

How we're making resources \textbf{accessible} at Cloud-SPAN:

\begin{itemize}
\tightlist
\item
  Our resources will be openly available, with no caveats, for use by those who cannot attend our workshops or who prefer self-led study.
\end{itemize}

\hypertarget{interoperable}{%
\subsection*{Interoperable}\label{interoperable}}
\addcontentsline{toc}{subsection}{Interoperable}

\textbf{Interoperable} means data/resources can be easily integrated with other data/resources, and be viewable in different programs, applications or workflows.

How we're making resources \textbf{interoperable} at Cloud-SPAN:

\begin{itemize}
\tightlist
\item
  We will supply data which are readable across different programs
\item
  We are providing resources in Markdown, meaning they should display in most browsers.
\end{itemize}

\hypertarget{reusable}{%
\subsection*{Reusable}\label{reusable}}
\addcontentsline{toc}{subsection}{Reusable}

\textbf{Reusable} is about making sure that data/resources are suitable for re-use in different settings.

How we're making resources \textbf{reusable} at Cloud-SPAN:

\begin{itemize}
\tightlist
\item
  We will be applying Creative Commons licenses to our resources so they can be reused and remixed by others.
\item
  We welcome (and encourage!) outside contributions of explanations and examples - see the \protect\hyperlink{ways-to-contribute}{Ways to contribute} for more information.
\end{itemize}

\hypertarget{forum}{%
\chapter{Cloud-SPAN Online Forum}\label{forum}}

There are two main places where you'll find the Cloud-SPAN community: GitHub and Peerboard.

To read more about how you can interact with us on GitHub, take a look at our \protect\hyperlink{ways-to-contribute}{Ways to Contribute} and our \href{https://github.com/Cloud-SPAN/CloudSPAN-handbook/blob/main/CONTRIBUTING.md}{GitHub Contribution Guide}.

For most questions and conversations, however, you'll probably want to use our \href{https://cloudspan.peerboard.com/}{Peerboard forum}. Here you can ask questions, pick people's brains and share any insights you've gained during or after one of our courses.
But the forum isn't just for problems. We'd also love to hear about your achievements and successful applications of what you've learned!

\hypertarget{signing-up-for-peerboard}{%
\section{Signing up for Peerboard}\label{signing-up-for-peerboard}}

Signing up for Peerboard is free, easy and only needs an email address. You should be able to view posts on the forum without making an account, but for the best experience possible we recommend signing up. This will allow you to create posts, join in with discussions and stay up-to-date with conversations.

\begin{enumerate}
\def\labelenumi{\arabic{enumi}.}
\tightlist
\item
  Visit \href{https://cloudspan.peerboard.com/}{cloudspan.peerboard.com} and click `Sign in' in the top right-hand corner.
\item
  If you would like to sign up with a Google, Facebook or LinkedIn account then choose one of these options and follow the instructions provided.
\item
  If you would prefer to make a new account then toggle over from `Log in' to `Sign up' and enter your email address and password.
\end{enumerate}

\begin{itemize}
\tightlist
\item
  You will be asked to confirm your email address via a confirmation email.
\end{itemize}

\begin{enumerate}
\def\labelenumi{\arabic{enumi}.}
\setcounter{enumi}{3}
\tightlist
\item
  After signing up you will be asked to enter your profile information. Don't forget to decide whether you want a daily digest of new posts delivered to your email inbox.
\end{enumerate}

That's it! You're all signed up! 🎉

  \bibliography{book.bib}

\end{document}
